%%%%%%%%%%%%%%%%%%%%%%%%%%%%%%%%%%%%%%%%%%%%%%%%%%
% Simple Drawing FrameWork User Guide
% wrtten by Nakaue Yoshito
% 2022/03/06
%%%%%%%%%%%%%%%%%%%%%%%%%%%%%%%%%%%%%%%%%%%%%%%%%%

\documentclass[a4paper, 11pt, oneside, onecolumn, openany]{jsarticle}

\usepackage[dvipdfmx]{graphicx}
\usepackage{here}
\usepackage{tabto}
\usepackage{listings}
\usepackage{jlisting}

\begin{document}
\NumTabs{20}

\newcommand{\function}[3]{
  \noindent
  \tab \texttt{#1} \\
  \tab \underline{引数} \tab #2 \\
  \tab \underline{返り値} \tab #3 \\\par
}

%%%%%%%%%%%%%%%%%%%%%%%%%%%%%%%%%%%%%%%%%%%%%%%%%%
% Setting for source code
%%%%%%%%%%%%%%%%%%%%%%%%%%%%%%%%%%%%%%%%%%%%%%%%%%
\lstset{
  basicstyle={\ttfamily},
  identifierstyle={\small},
  commentstyle={\smallitshape},
  keywordstyle={\small\bfseries},
  ndkeywordstyle={\small},
  stringstyle={\small\ttfamily},
  frame={tb},
  breaklines=true,
  columns=[l]{fullflexible},
  numbers=left,
  xrightmargin=0zw,
  xleftmargin=3zw,
  numberstyle={\scriptsize},
  stepnumber=1,
  numbersep=1zw,
  lineskip=-0.5ex
}

%%%%%%%%%%%%%%%%%%%%%%%%%%%%%%%%%%%%%%%%%%%%%%%%%%
% Title
%%%%%%%%%%%%%%%%%%%%%%%%%%%%%%%%%%%%%%%%%%%%%%%%%%
\title{\vspace{-3cm}Simple Drawing FrameWork}
\author{Y.Nakaue}
\maketitle


%%%%%%%%%%%%%%%%%%%%%%%%%%%%%%%%%%%%%%%%%%%%%%%%%%
% プログラムの開始と終了
%%%%%%%%%%%%%%%%%%%%%%%%%%%%%%%%%%%%%%%%%%%%%%%%%%
\section{プログラムの開始と終了}
\subsection{ライブラリ機能の初期化}
\function{void sdfw\_init()}{無し}{無し}

\subsection{ライブラリ機能の終了処理}
\function{void sdfw\_quit()}{無し}{無し}


%%%%%%%%%%%%%%%%%%%%%%%%%%%%%%%%%%%%%%%%%%%%%%%%%%
% メインループ
%%%%%%%%%%%%%%%%%%%%%%%%%%%%%%%%%%%%%%%%%%%%%%%%%%
\section{メインループ}
\subsection{描画内容の更新}
\function{bool update}{無し}{メインループの更新可否}
描画内容を最新の状態に更新する.常にtrueを返す.\\


%%%%%%%%%%%%%%%%%%%%%%%%%%%%%%%%%%%%%%%%%%%%%%%%%%
% 描画のための設定
%%%%%%%%%%%%%%%%%%%%%%%%%%%%%%%%%%%%%%%%%%%%%%%%%%
\section{描画のための設定}
\subsection{ウィンドウの作成}
\function{int open\_window(width, height)}{width: 横幅,height: 高さ}{作成したウィンドウID}
描画を行うためのウィンドウを作成し,画面前面に表示する.引数には作成するウィンドウの横幅と高さを指定する.1つのプログラムの中で複数のウィンドウを作成することが可能で,返り値として作成したウィンドウに割り当てられたウィンドウIDを返す.ウィンドウIDは,作成した順に0からの連番で整数値が返される.


%%%%%%%%%%%%%%%%%%%%%%%%%%%%%%%%%%%%%%%%%%%%%%%%%%
% 主な描画関数
%%%%%%%%%%%%%%%%%%%%%%%%%%%%%%%%%%%%%%%%%%%%%%%%%%
\section{主な描画関数}

\subsection{文字列}
\function{void print\_text(text, win=0)}{text: 文字列データ,win: 描画先ウィンドウID}{無し}
引数に指定した文字列を描画する.第2引数に描画を行うウィンドウIDを指定でき,デフォルトではメインウィンドウへの描画を行う.1フレーム内で複数回この関数を呼び出すと自動的に改行され,2回目以降の呼び出し時には次の行に出力される.


\end{document}